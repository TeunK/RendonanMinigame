\documentclass[12pt]{report}
\usepackage{graphicx}
\usepackage{fancyhdr}
\usepackage{hyperref}
\usepackage{wrapfig}
\usepackage{url}
%\usepackage[left=20mm,top=20mm,bottom=20mm,right=20mm]{geometry}
\author{Teun Kokke, s1242775}

\begin{document}



\begin{titlepage}

\end{titlepage}

\title
{
	\vspace*{\dimexpr-4.4in-\topmargin-\headsep-\headheight-\baselineskip}%
	\hspace*{\dimexpr-1.2in-\evensidemargin-\parindent}%
	\makebox[\paperwidth][r]{\includegraphics[height=3cm]{logo.png}}
    Software Engineering Large Practical
}
\maketitle

\section*{General Concept}
My project is the combination of a website, database and a turn-based game, with the game itself at its core. Users are able to register an account on the website, requiring only a username and password, and can then immediately jump in to play the game.\\
Players start with a basic set of stats: \textit{experience, money, health, strength and agility}. From this point the player is expected to fight the monsters that will continuously re-spawn upon death. By killing the monsters, the player earns both money and experience, where the money can be used to upgrade stat points, making the player stronger and able to combat harder, more rewarding monsters. The experience plays a major role in the user ranking and contributes to leveling up the user, improving his looks.\\
When the player leaves the game, the game progress will be saved and re-loaded when he/she returns. However, when the player dies, all progress will be lost. Only at this point, the stats will be stored in the highscores.\\
The highscores can be viewed on the website, and contain the stats of each death that occurred for players. The ranks are decided stat-wise, each stat can be selected and filtered individually in ascending/descending order. Additionally, each player can be sought for manually, in the search-bar.
\vspace{2em}
\begin{quote}
\textit{In attempt to explain the code modules behind the project, a "User Guide" section is included to this report for both the website and game. This guide displays the result of the code, and can thus be referred back to in the module section, making it more clear what the code is there for, and what it really does.}
\end{quote}
\pagebreak

\section*{Structure}
\subsection*{Website}
\subsubsection{Design}
Coming up with the following design was pretty straight-forward. In order to play a game, as a specific user, and be able to see the highscores, on a website, the website is divided into 4 (user-interact-able) pages: Main, Game, Highscore and Registration.
\begin{enumerate}
\item Starting from the \emph{\textbf{main page}}, a few things can be noticed:
\begin{enumerate}
\item There is a menubar on top of the website which is present at any of the pages in the website.
\begin{itemize}
\item On the left the text displaying "Rendonan" is a link sending the user back to the main page.
\item On the right, either a login form is displayed when the user is not logged in, or the username of the logged-in user along with a link to logout or delete his/her account.
\end{itemize}
\item There are 3 links in the main content section: Play, Highscores, Register.
\item There is a simplified highscore table displayed on the right.
\end{enumerate}
\item From the main-page, by pressing the url to \emph{\textbf{register}} an account the user will be forwarded to the registration page. Here, if the user was already logged in, he/she will be notified of this. Otherwise 2 forms will display, one to register, one to login. Both forms only accept alphanumeric strings between length 4 and 20 characters.\\
Upon successful registration, the user is immediately logged in with the given username and redirected back to the main page.
\item From the main-page, the \emph{\textbf{highscores}} link can be clicked. These scores are primarily ordered by score and then by experience (in descending order). But it is possible to filter the results based on purely experience/wealth/health/strength/agility in both ascending and descending order by pressing the '\textasciicircum' or 'v' buttons.\\
Additionally, specific usernames can be sought for using the user-search-bar right above the highscore table.
\item From the main-page, the \emph{\textbf{play game}} link can be clicked to open the game page.
\begin{enumerate}
\item If the user was not logged in, he/she will be immediately redirected to the register page, implying that he/she must first register and then try again.
\item If the user was logged in, the game will automatically start loading and running.
\begin{itemize}
\item When the game starts, the user's data (user that is logged in) will be queried from the database and then received by the game. The user will be held in a waiting room until this data was succesfully received by the game. Otherwise the game client will repeatedly attempt to establish a connection.
\item Once the game has successfully started up and loaded the user data, the game will start and allow the user to play.
\item The user's progress will be automatically saved to the database. This event occurs every 5 seconds. Therefore when the player stops playing the game, he/she can proceed where he/she left when returning to the game in the future.
\item When the player dies, the progress is sent to the highscores, and will then be reset for the user such that the player has to restart the game completely.\\All that remains from his past game is the score, which (depending on how good was relative to the other scores) will be visible in the highscore page.
\end{itemize}
\end{enumerate}
\end{enumerate}
Symfony follows a systematic approach to rendering html responses, which in simplified form looks as follows:

\begin{figure}[h]
\centering
\makebox[\textwidth]{ \includegraphics[scale=0.33]{symfonySystem.jpg}}
\caption{Displays the back-end path from a http request to generating the appropriate response.}
\end{figure}
Following this structure, the controllers take care for generating the required output, with each of the requirements being mentioned in the points above. Aside of the pages that are visibly rendered for the user, some actions contain pure logic blocks, e.g. user logout and game-load/save. These will execute the appropriate logic commands (typically session and/or database related) and will then redirect to a different page.


\subsubsection{Refactoring}
As for refactoring the website, little refactoring was required. This is because I did not have to write many re-usable "modules". The main task here was to set-up the routing in order to direct the right url to the right controller with the right action (function). The code used inside the controllers was generally quite unique for each page, as each page is indeed unique on its own. The only common factor is the user-session, for which I created a separate class. This class contains a method \emph{get\_sessionData()} which returns an array containing all the useful user data.\\\\
If at any point I wanted to store a new piece of information about the user, only this class would have to be modified, and then every page would immediately have access to it.\\

\subsection*{Game}
\subsubsection{Design}
Similar to the website, the game is expected to log a user in, allow the user to play, stop the game when the user dies, and be tested by the developer when required. Therefore it is separated in 4 individual rooms / views: Connect, Game, Death and Test.
\begin{enumerate}
\item In the \emph{\textbf{connection room}}, the controller object is instantiated, which controls the entire gameflow. At this point, its main tasks are to check whether the game should run in game-mode, runtime-test mode or unit-test mode.
\begin{enumerate}
\item If the unit-test mode was set, the controller will immediately forward the game to the test-room.
\item Otherwise, the controller will attempt to connect to the database in order to fetch the user stats. Once the user stats were received (using \textsl{http\_get()}), it will forward the game to the game-room.
\end{enumerate} 
\item In the \emph{\textbf{game room}}, 
\begin{enumerate}
\item The controller: 
\begin{itemize}
\item creates stat-upgrade purchase-buttons for each of the upgradable stats
\item creates the player instance
\item respawns the monsters: \textsl{monster\_respawn()}
\item draws the user interface: \textsl{draw\_UI()} (stats menu)
\item draws the combat menu: \textsl{draw\_menu()}
\item tracks the user stats
\item controls game event timers
\end{itemize}  
\item The stat-upgrade purchase buttons use the player stats (stored in the controller) to decide the cost of the upgrade, the amount by which the stat is upgraded and does the arithmetic that comes to play when pressed.
\item The player instance draws the player as entity with appropriate image based on the level which it calculates using \textsl{calc\_level()} after synchronizing with the stats from the controller using \textsl{player\_update()}
\item The monster instances 
\begin{itemize}
\item Draw themselves as entity with appropriate image based on level (set randomly by a user-defined range in the combat menu upon creation by the controller).
\item Calculate their own combat stats based on their level
\item Create a message upon death, incrementing the controller's xp and money stats (\textsl{monster\_death()}).
\end{itemize}
\item the user interface displays the stats of the player stored in the controller object.
\item The combat menu
\begin{itemize}
\item Displays a charging attack-bar based on the controllers attack-cooldown timers (influenced by the agility stat), which when filled allows the player to attack.
\item Draws an \textsl{attack()} button when not in auto-combat mode, which can be pressed in order to attack the monster.
\item Draws an auto-attack button which toggles the auto-combat mode variable in the controller. When auto-combat mode is on, the attack-bar charges slower, but the attack button will disappear as the player will automatically \textsl{attack()} the monster when the bar is filled instead.
\item Draws a potion button which can be pressed to heal the player slightly at the cost of money.
\item Draws a \textsl{monster\_controller()} which includes in-/decrement buttons for the future monster's re-spawn-level range.
\item Draws a special-skill (\textsl{draw\_sunflash()}) charge bar which when filled displays an activation button, which in turn triggers a particle system object that damages the monster.
\end{itemize}
Due to the complexity, here is a visual representation showing the game room:

\begin{figure}[h]
\centering
\makebox[\textwidth]{ \includegraphics[scale=0.33]{userGuide.jpg}}
\caption{Visual representation of the game room, displaying many of the modules mentioned above.}
\end{figure}

\end{enumerate}
\item In the \emph{\textbf{Death room}}, the controller draws a simple message notifying the user of his/her death and waits for a user-input response to restart the game.
\item In the \emph{\textbf{Test room}}, the controller instantiates a test-object \emph{obj\_TestConnection}, which:
\begin{enumerate}
\item Executes all unit tests for the functions used throughout the game (\emph{test\_exec()})
\item Tests if it can connect to the server (php controllers)
\item Tests if it is successful in its attempt to load user data
\end{enumerate}
\end{enumerate}
\subsubsection{Refactoring}
Little "RE-factoring" was required as I have spent a lot of time working on projects in Gamemaker in the past, having given me the experience and knowledge of many potential threats and issues that may be lurking in the near (and even distant) future. Therefore many of these issues were (mostly unconsciously) treated before they became an issue.\\
It should indeed have become apparent that the controller object sets the entire game in motion. Every instance is responsible for only itself, and is at most dependent on the controller, of which only exactly 1 instance exists at any time, meaning that this dependency is harmless.\\
Due to this design, it is unlikely that modifying any module breaks any other module (as long as I (or whichever programmer) sticks to the rules of the module in question, meaning that for example the return type should not change).\\

\subsubsection{Function Structure}
Since Gamemaker builds an xml-like structure in order to organise the functions in sub-folders, but on the harddrive only stores the scriptnames in one single folder, here is an image the structure:

\begin{figure}[h]
\makebox[\textwidth]{\includegraphics[scale=0.52]{script_hierachy.jpg}}
\caption{Script Hierachy, displaying the alignment of scripts used in gamemaker. Their content is located in the folder GM\_SOURCE/sourcecode/rendonan\_release.gmx/scripts.}
\end{figure}
\vspace{2em}
\section*{Difficulties}
Before the development started (around the time of writing the proposals) I successfully predicted to get stuck for a relatively long period of time on exactly 3 points:
\begin{itemize}
\item Since we were tasked to behave like real "Software Engineers" as opposed to simple "programmers", I took it on me to learn using an industrially accepted framework, "Symfony", for the website. In the past I succeeded to teach myself how to program websites (including back-end, registration / login system etc.), starting blank and developing the entire set-up and structure. It was complicated using the framework as I was not "allowed" to use the majority of the methods I was used to using in the past. For this reason, it is possible that I have (unconsciously) written functions for tasks that Symfony could have solved instead. However for obvious reasons, I had to find a proper balance between "spending development time" and "spending research time".\\
The research was time-spent with no visible progress, yet, having awareness about getting stuck at many points during the development due to the lack of knowledge is something I did not look forward to.

\item The user log-in system of Symfony is completely different to what I am used to working with. From my past experience, logging in a user was as easy as setting session-variables (id, username, whichever). Symfony however, appears to use some sort of firewall security system which tracks each user. I have spent days in attempt to understand how this works, tried many different suggested setups but all failed. The clock kept ticking so in the end I gave up trying to cooperate with the framework on this point.\\
Therefore, plan B, I used "sessions" directly (or rather, a Symfony-accepted version of them). This method worked, although it did mean that users are not truly recognised as "users" by the framework. Although I designed my own system for this in a way that enabled me to still have the required control over the users.

\item Connecting GM to the Database using http\_get is something I have been stuck on for a long time in the past. It was one of the main reasons why I have never made browser games, but rather download-able clients which allow the use of dll files and c++ socketing (allowing even multiplayer functionality).\\
It took me several days to solve this issue, having created a separate project just for the sake of experimenting with http requests. It required me to run a software update for Gamemaker, after having purchased an additional module for the software which allowed me to include external JS scripts. With such JS script, I triggered an asynchronous Ajax call to the website (url defined in the routing.yml, which then loaded the desired (php) controller), which would query the database and return the results back to the game. 

\end{itemize}

\section*{Testing}
\subsection*{Website}
My apologies for waiting so long for researching on how to do testing in a php framework. This was partially because due to lack of imagination on what can be tested, as well as the fact that, the lack of understanding on how the framework itself works, added sufficient complication on its own to not immediately demand for further research on how this could then also be tested. However if I were to add a mark to the "bus-factor" before my tests were written, this would have been a bad excuse for making a high-end company lose revenue.\\
On a positive note, I did succeed in installing a testing unit called "phpunit", with which I wrote the tests for the following:
\begin{itemize}
\item each of the pages were tested to load correctly
\item When no page was specified by the user, he/she will automatically be redirected to the main page
\item The registration and login form validations were tested for "bad" input (e.g. quotes, semi-colons, unaccepted string-length, etc.).
\item The error-messages returned by the form validation
\item Successful registration tested to indeed be successful
\item Account deletion tested to indeed delete the account
\item Adding a user highscore was tested using a long sequence of tests, starting from registering the user, logging it in, saving new game data, letting it "die" to store the data into the highscores, ensuring the game score was reset, (removing the highscore entry again), and removing the user.
\item The load and save-game scripts were tested which the game will be using to access in order to load and store data onto the server.
\end{itemize}
\subsection*{Game}
Tests were also written for parts of the game engine (front-end). Since Gamemaker does not come with any testing modules, i created my own, and have split them up into 4 sections:
\begin{enumerate}
\item \emph{\textbf{Type Checks:}} Since I used many functions, each function call has been embedded with type-checks to ensure that the input contains the correct type, making sure that when calling a function it has been given proper arguments. Since GML is a dynamically typed language, not knowing a function has been given a wrong argument type may be difficult to spot.
\item \emph{\textbf{Logic tests:}} From my personal experience, logic errors are the most commonly occurring, most difficult to spot and therefore also most difficult to fix. The best method of testing if the game logics work correctly is generally to actually run the game and see if everything happens the way it is expected to.\\
However, I did create some methods that display in-game variables (that would otherwise be hidden). Being able to see the actual values of these variables is one of the fastest ways to spot if something is wrong.\\
Most potential for these errors lies in event triggers and erroneous timers. So these are displayed when the test-mode is active.
\item \emph{\textbf{Unit tests:}} Each function that takes input and has a return-value can be tested with unit tests. Therefore each function that matches these criteria is considered with unit tests.
\item \emph{\textbf{Connection tests:}} A more complex type of test, but also one of the more important one. Since the game is expected to connect to the database, tests are written that confirm that the game indeed manages to successfully connect to the server, and manages to receive user data.
\end{enumerate}
I have created my own testing module to execute the latter 2 testing types. This will trigger when the test\_mode in the controller object is set to unit-testing. At this point, all of the unit tests will be executed, as well as the connection to the server.\\
If at point in time, any of the tests mentioned above fails when either in test-mode or unit-test mode, a warning message will be displayed.\\
The controller object also contains an "active-testing" variable (which can be toggled by pressing spacebar), while in testing-mode. When active, even tests that passed will return a message, confirming that the test was executed and passed.\\
If the game is in regular game-mode, of course none of these messages will pop-up. The game would execute as if nothing happened.
\subsection*{Test Coverage}
Of course everything could be tested in much greater depth, both the website and the game. However for the level of this project, the amount of testing done roughly matches the requirement.\\
One of the more immediate things that could be tested is testing the game-save script from the game's point of view (as opposed to what is currently only done from the website's perspective). This would require the POST-request sent by the game, to also expect to receive a return value about whether or not the request was received successfully. The same concept applies for storing the user's score into the highscores table.\\
Another (very interesting) test would be a complete system test, not only covering either the website or game separately, but both at the same time. Registering a user-account, making it log in, start-up the game, let the game execute in an automated-mode, expecting it to save the score (check if this works), expect the bot to die at some point and see if the score was saved, then going to the highscore page to confirm the score is there, and then delete the score and the user account.\\
This system would cover the typical lifetime of a standard user. Meaning that if this test passes, it can be expected that users following the same pattern won't have any issues either. The only problem is that writing this test would involve the server sending an additional value to the game stating that it should run in automated mode. Although most of this is indeed already done as the player can enter auto-attack mode, the point of converting from web-platform testing to game-testing (and back) adds complication as intuitively neither of the elements actually know of eachothers existence.

\section*{Deficiencies}
I do not believe I have made any large "errors" in my decisions when planning out the setup of the project elements. All in all I feel like I have followed a rather smooth and consistent workflow without large drawbacks.\\
The biggest issues I have stumbled upon are, again, the lack of knowledge on the Symfony platform from my part. Although it is arguable whether this is due to a "bad decision" or an expected side-effect of a choice to learn industrially-accepted software engineering standards for PHP development.\\
Either way, the deficiencies I am most aware of are:
\begin{itemize}
\item Having repeatedly failed to use Symfony's methods of generalising database queries, referred to as "entity repositories". I had made many different attempts, used forums, read the manual etc. in order to figure out how these repositories were supposed to be used. I managed to use them for standard queries, but in order to give it arguments I had to re-write the entire query with just that argument being different. Therefore after a long struggle for several times I gave up and decided to run the queries directly inside the controllers that rendered the pages (queries for fetching highscore data and user data).
\item As mentioned previously, the user-login system for the Symfony inbuilt-system was another major drawback as I made many attempts but still could not get this to work. Also here I was forced to fall back to my old knowledge of using sessions.
\item In Gamemaker, it is not possible to define a function inside a script. Instead, it expects the user to create a new individual function script, and the name of the file will be the name of the function. Arguments passed will be referred to as "argumentx" (with x being any number from 0 up to 19). For an engineer that is used to programming in languages such as Java and C++ this may seem weird at first.
\end{itemize}

\section*{Future Improvements}
\subsection*{Choice of technology / language}
The reason why I like php is due to its simplicity. It does not require any compiling and works on almost any webserver. However I learnt that using a framework such as Symfony removed the simplicity, and it is no longer supported on any webserver either. Therefore using php does not have any more value when planning to build relatively large and stable web platforms.\\
Writing plain JavaScript, especially complete games, can become messy very easily. However using Gamemaker solves this issue perfectly.
\subsection*{Future project potentials}
There are many ways in which this project can improve, and at high speed. I have attempted to follow an as solid implementation as possible, enabling the ability to build new elements on top of this foundation very quickly and easily.\\
The game can include items that the user can purchase, an inventory system can be added, monsters may get special skills, requiring the player to pay more close attention to the game during play, etc.\\
The website could become much more interactive, by adding visual and technical improvements. Aside of having plain highscores, it could be possible to compare user stats.\\
Perhaps it is possible to make the game multiplayer over the network entirely, although this would require further research in order to find out how to solve delay and server/client-firewall issues (especially when using a UDP connection), to which every browser may respond differently.

\section*{Conclusion}
It is safe to say that by the time I got accepted to Edinburgh University, I had already spent well over 1000 hours programming in Gamemaker. This comes both with up and down-sides. No-one has taught me how to work with it, I did all the experimenting on my own and have thus far come to the conclusion that the structure of building-up projects in the order I have used for this one is the most efficient and requires least re-factoring. By "this structure" I mean to say: creating a controller object and let it take care of everything. Let it store all the important variables of the user, create instances on demand, and allow these instances to depend on the controller if required. But not make the controller dependent of anything else.\\
Like writing a book, every new feature (be it a menu, chatbox, attack event, mathematical formula to calculate a stat) should be written in one piece, on one location as separate function, preferably being completely independent of everything (other than the input arguments it received).\\
This way, it is possible to "zoom in" on each part (/module), and be able to edit it without having to worry about anything else breaking down.\\
Having this experience, it allows me to very easily and quickly create new projects without having to think of it too much. Also, as I manage to stay consistent in my system, I don't have to change my mind as often, meaning that when refactoring becomes truly necessary most often the best solution is to actually rewrite an entire function. Luckily this rarely happens.\\
On the other hand, I developed the system based on my own experiences. Other people may have had other experiences and therefore have a completely different system. I may become blind to certain issues as I don't consider them "issues" anymore: I have become used to them and have my own ways of dealing with them, whereas other programmers may not find it as obvious. This is an issue that heavily relates to software engineering.

\end{document}
